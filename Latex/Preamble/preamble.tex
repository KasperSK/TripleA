
%%%%%%%%%%%%%%%%%%%%%%%Opsætning af format%%%%%%%%%%%%%%%%%%%%%%%%%%
\documentclass[a4paper,twoside]{memoir} %A4papir, to side, størrelse 12, type memoir 	%TWOSIDE!!!!!!	
%\documentclass[a4paper,oneside]{memoir} %A4papir, to side, størrelse 12, type memoir 	%ONESIDE!!!!!!

% For dansk opsætning med æ, ø og å, samt pænere orddeling.
\usepackage[utf8]{inputenc}				% æøå
\usepackage[danish]{babel}				% dansk opsætning
\renewcommand{\danishhyphenmins}{22}	% fikser babel fejl/bedre orddeling
\usepackage[T1]{fontenc}
\usepackage{lmodern} 
%%Høre til under tabel (Pakker, men skal stå før pgfplot pga. default værdier)%%
\usepackage[table]{xcolor}	
\usepackage{ragged2e}	

%%%%%%%%%%%%%%%%%%%%%%%%%%%%%%%%%%%%%%%%%%%%%%%%%%%%%%%%%%%%%%%%%%%%%%%%%%%%%%%%

\setcounter{tocdepth}{2} % inkludere sub + subsubsection i inholdsfortegnelde
\setsecnumdepth{subsection}

\usepackage{titlesec} 
\titleformat{\chapter}		  % Fjerne Kapitel og tal fra chapters
			{\Large\bfseries} % format
			{}                % label
			{0pt}             % sep
			{\huge}           % before-code
% til mellemrum

\titlespacing\section{0pt}
{24pt plus 4pt minus 2pt}{6pt plus 2pt minus 2pt} %halvt mellemrum efter section
\titlespacing\subsection{0pt}
{18pt plus 4pt minus 2pt}{2pt plus 2pt minus 2pt} %kun lidt mellemrum efter subsection
\titlespacing\subsubsection{0pt}
{18pt plus 4pt minus 2pt}{2pt plus 2pt minus 2pt} %kun lidt mellemrum efter subsubsection 

\setlength\parindent{0pt} %Ingen indryk efter ny afsnit

\usepackage[bottom]{footmisc} %fodnoter i bunden af arket

%Marginer indstilles
\raggedbottom	%Ingen side strech for twopage
\setlrmarginsandblock{3cm}{3cm}{*}		%Højre - venstre
\setulmarginsandblock{3cm}{2.5cm}{*}	%Øverst - nederst
\checkandfixthelayout[nearest]    		%Specifikt valg af højde algoritme
\usepackage{ragged2e,anyfontsize}		% Justering af elementer
\usepackage{fixltx2e}					% Retter forskellige fejl i LaTeX-kernen

% Sidehoved og -fod
\let\footruleskip\undefined  %fixer memoir default footruleskip
\usepackage{fancyhdr}
\pagestyle{fancy}
\fancyhf{}
\fancyhead[C]{\textit{Automatisk Vandingssystem}}
\fancyfoot[RO,LE]{\thepage\ af \thelastpage}  	% sættet sidetal tal h/v efter om der er ulige
												% eller lige sidetal
\fancypagestyle{plain}{% bruges ved Undtagelser						
  \fancyhf{}%
  \fancyfoot[RO,LE]{\thepage\ af \thelastpage}	
  \renewcommand{\headrulewidth}{0pt}			%Ingen linje ved chapter, kun sidetal
}
%%%%%%%%%%%%%%%%%%%%%%%%%%%%%%%%%%%%%%%%%%%%%%%%%%%%%%%%%%%%%%%%%%%

%%%%%%%%%%%%%%%%%%%%Pakker og design til forside%%%%%%%%%%%%%%%%%%%
\usepackage{amsmath}
\usepackage{tikz}
\usepackage{epigraph}
% Tilføjet af Karsten. Bruges til at lave altive http link
\usepackage{hyperref}

\renewcommand\epigraphflush{flushright}
\renewcommand\epigraphsize{\normalsize}
\setlength\epigraphwidth{1\textwidth}

\definecolor{titlepagecolor}{cmyk}{1,.60,0,.40}

\DeclareFixedFont{\titlefont}{T1}{ppl}{b}{it}{0.5in}




% The following code is borrowed from: http://tex.stackexchange.com/a/86310/10898

\newcommand\titlepagedecoration{%
\begin{tikzpicture}[remember picture,overlay,shorten >= -10pt]

\coordinate (aux1) at ([yshift=-15pt]current page.north east);
\coordinate (aux2) at ([yshift=-410pt]current page.north east);
\coordinate (aux3) at ([xshift=-4.5cm]current page.north east);
\coordinate (aux4) at ([yshift=-150pt]current page.north east);

\begin{scope}[titlepagecolor!40,line width=12pt,rounded corners=12pt]
\draw
  (aux1) -- coordinate (a)
  ++(225:5) --
  ++(-45:5.1) coordinate (b);
\draw[shorten <= -10pt]
  (aux3) --
  (a) --
  (aux1);
\draw[opacity=0.6,titlepagecolor,shorten <= -10pt]
  (b) --
  ++(225:2.2) --
  ++(-45:2.2);
\end{scope}
\draw[titlepagecolor,line width=8pt,rounded corners=8pt,shorten <= -10pt]
  (aux4) --
  ++(225:0.8) --
  ++(-45:0.8);
\begin{scope}[titlepagecolor!70,line width=6pt,rounded corners=8pt]
\draw[shorten <= -10pt]
  (aux2) --
  ++(225:3) coordinate[pos=0.45] (c) --
  ++(-45:3.1);
\draw
  (aux2) --
  (c) --
  ++(135:2.5) --
  ++(45:2.5) --
  ++(-45:2.5) coordinate[pos=0.3] (d);   
\draw 
  (d) -- +(45:1);
\end{scope}
\end{tikzpicture} }
%%%%%%%%%%%%%%%%%%%%%%%%%%%%%%%%%%%%%%%%%%%%%%%%%%%%%%%%%%%%%%%%%%%%

%%%%%%%%%%%%%%%%%%%%%%%%%%%%%%Pakker%%%%%%%%%%%%%%%%%%%%%%%%%%%%%%%%
\usepackage{graphicx} 				% Haandtering af eksterne billeder (JPG, PNG, EPS, PDF)
\usepackage{multirow}               % Fletning af raekker og kolonner (\multicolumn og \multirow)

\usepackage{wrapfig}				%for figure der skal have tekst om sig
\usepackage{float}					% Muliggoer eksakt placering af floats, f.eks. \begin{figure}[H]

\usepackage{enumerate}				% Muliggoer at man kan bruge fx a) i enumerate

\usepackage{pdflscape}				% Muligør landskab på enkelte sider
\usepackage{tabularx}				% Tabeller med X width

\usepackage{../Latex/Usecases/usecases}				% Usecase

\usepackage{mathtools}				% Formler og matematik

\usepackage{caption}
%\usepackage{subcaption}			% This package kills me :/

\usepackage{pdfpages}				% til at inkludere PDF filer

\usepackage{gensymb}				% symboler til latex

% For at holde styr på mangler i teksten
\usepackage[footnote,draft,danish,silent,nomargin]{fixme}	
%%%%%%%%%%%%%%%%%%%%%%%%%%%%%%%%%%%%%%%%%%%%%%%%%%%%%%%%%%%%%%%%%%%%

% If then
\usepackage{ifthen}

% ¤¤ Litteraturlisten ¤¤ %
\usepackage[danish]{varioref}				% Muliggoer bl.a. krydshenvisninger med sidetal (\vref)
\usepackage{natbib}							% Udvidelse med naturvidenskabelige citationsmodeller
\bibpunct[,]{[}{]}{;}{a}{,}{,} 				% Definerer de 6 parametre ved Harvard henvisning 
											% (bl.a. parantestype og seperatortegn)
\bibliographystyle{../Latex/Litteratur/harvard}		% Udseende af litteraturlisten.
\usepackage{hyperref}

% Ordliste
\usepackage[nonumberlist,toc]{glossaries}
\makeglossaries
\newglossaryentry{betalingskort}{
	name=Betalingskort,
	description={Et kort som kan bruges som betalingsmiddel f.eks. et Dankort eller VISA/Dankort}
}

\newglossaryentry{betalingsterminal}{
	name=Betalingsterminal,
	description={En terminal der bruges til at udføre transaktioner, når et betalingskort bruges som betalingsmiddel}
}

\newglossaryentry{vare}{
	name=Vare,
	description={En vare kan bruges til at omtale en alkoholisk drik eller en øl}
}

\newglossaryentry{system}{
	name=Systemet,
	description={Katrines Kælders Kasseapperat}
}

\newglossaryentry{bartender}{
	name=Bartender,
	description={Den primære bruger af systemet}
}

\newglossaryentry{Administrator}{
	name=Admin,
	description={Brugeren som varetager systemet}
}

\newglossaryentry{gls-MVVM}{
	name=Model-View-ViewModel,
	description={En systemarkitektur mønster som bruges i brugergrænsefladen}
}
\newacronym{MVVM}{MVVM}{Model-View-ViewModel\glsadd{gls-MVVM}}

\newglossaryentry{brugergraenseflade}{
	name=brugergrænseflade,
	description={Grænseflade mellem bruger og system}
}
\newglossaryentry{gls-GUI}{
	name=Graphical User Interface,
	description={Se \Gls{brugergraenseflade}}
}
\newacronym{GUI}{GUI}{Graphical User Interface\glsadd{gls-GUI}}

\newglossaryentry{gls-WPF}{
	name=Windows Presentation Framework,
	description={Et framework der bruges til at præsentere brugergrænsefladen}
}
\newacronym{WPF}{WPF}{Windows Presentation Framework\glsadd{gls-WPF}}

\newglossaryentry{gls-XAML}{
	name=Extensible Application Markup Language,
	description={Et sprog der bruges til at deklarere brugergrænsefladen}
}
\newacronym{XAML}{XAML}{Extensible Application Markup Language\glsadd{gls-XAML}}

\newglossaryentry{gls-EF}{
	name=Entity Framework,
	description={Et framework der bruges til at tilgå databasen}
}
\newacronym{EF}{EF}{Entity Framework\glsadd{gls-EF}}

\newglossaryentry{gls-DAL}{
	name=Data Access Layer,
	description={En grænseflade mellem Entity Framework (databasen) og systemet}
}
\newacronym{DAL}{DAL}{Data Access Layer\glsadd{gls-DAL}}

\newglossaryentry{forretningslogik}{
	name=forretningslogik,
	description={Et reglsæt for systemets behandling af data}
}
\newglossaryentry{gls-BLL}{
	name=Business Logic Layer,
	description={Se \Gls{forretningslogik}}
}
\newacronym{BLL}{BLL}{Business Logic Layer\glsadd{gls-BLL}}

\newglossaryentry{View}{
	name=View,
	description={En grafisk brugergrænseflade}
}

\newglossaryentry{ViewModel}{
	name=ViewModel,
	description={En abstraktion mellem forretningslogikken og den grafiske brugergrænseflade}
}

\newglossaryentry{Model}{
	name=Model,
	description={Se \Gls{forretningslogik}}
}

\newglossaryentry{FluentAPI}{
	name=Fluent API,
	description={En API som i forhold til databasen binder objekt relationer til entity relationer}
}

\newglossaryentry{gls-API}{
	name=Application Programming Interface,
	description={En dokumenteret grænseflade til software komponenter}
}
\newacronym{API}{API}{Application Programming Interface\glsadd{gls-API}}

\newglossaryentry{gls-DTO}{
	name=Data Transfer Object,
	description={En datamæssig afkobling af system og database, hvor ens databasemodel overføres til et system objekt}
}
\newacronym{DTO}{DTO}{Data Transfer Object\glsadd{gls-DTO}}

\newglossaryentry{gls-REST}{
	name=Representational State Transfer,
	description={Et arkitekturmønster til at designe netværksapplikationer}
}
\newacronym{REST}{REST}{Representational State Transfer\glsadd{gls-REST}}

\newglossaryentry{ASPNET}{
	name=ASP.NET,
	description={Et web applikations framework}
}

\newglossaryentry{gls-Dao}{
	name=Data Access Object,
	description={En klasse som abstrahere \gls{DAL} i forhold til det emne, der omhandles}
}
\newacronym{Dao}{Dao}{Data Access Object\glsadd{gls-Dao}}

\newglossaryentry{controller}{
	name=controller,
	description={En klasse der styrer logikken for det emne, der omhandles}
}

\newglossaryentry{RESTful}{
	name=RESTful,
	description={Et API som opfylder \gls{REST} mønsteret}
}

\newglossaryentry{WebAPI}{
	name=WebAPI,
	description={En implementering af et \gls{RESTful} API}
}

\newglossaryentry{gls-DSD}{
	name=Data Structure Diagram,
	description={En diagram som beskriver data strukturer især i forhold til databasen}
}
\newacronym{DSD}{DSD}{Data Structure Diagram\glsadd{gls-DSD}}

\newglossaryentry{DDS-Lite}{
	name=DDS-Lite,
	description={Et program som bruges til at designe databaser}
}

\newacronym{IHA}{IHA}{Ingeniørhøjskolen Aarhus Universitet}
\newacronym{ASE}{ASE}{Aarhus University School of Engineering}




\newcommand{\systemBDD}[3][FLAF]{
\ifthenelse{ \equal{#1}{FLAF} }{
\begin{figure}[H]
	\centering
	\includegraphics[width=#2\textwidth]{Systemarkitektur/#3/#3_BDD.png}
	\caption{Block Definition Diagram af #3}
	\label{fig:#3_BDD}
\end{figure}
}{
\begin{figure}[H]
	\centering
	\includegraphics[width=#2\textwidth]{Systemarkitektur/#1/#1_BDD.png}
	\caption{Block Definition Diagram af #3}
	\label{fig:#1_BDD}
\end{figure}
}

}

\newcommand{\systemIBD}[3]{
\begin{figure}[H]
	\centering
	\includegraphics[width=#1\textwidth]{Systemarkitektur/#2/#2_IBD.png}
	\caption{Internal Block Diagram af #3}
	\label{fig:#2_IBD}
\end{figure}
}

\newcommand{\systemDomainModel}[3]{
\begin{figure}[H]
	\centering
	\includegraphics[width=#1\textwidth]{Systemarkitektur/#2/#2_Domain_Model.png}
	\caption{Domænemodel af #3}
	\label{fig:#2_Domain_Model}
\end{figure}
}

\newcommand{\systemAllokeringsDiagram}[3]{
\begin{figure}[H]
	\centering
	\includegraphics[width=#1\textwidth]{Systemarkitektur/#2/#2_Allokeringsdiagram.png}
	\caption{Allokeringsdiagram af #3}
	\label{fig:#2_Allokeringsdiagram}
\end{figure}
}

\newcommand{\PakkeDiagram}[3]{
\begin{figure}[H]
	\centering
	\includegraphics[width=#1\textwidth]{SoftwareArkitektur/#2/Diagrammer/#3_PakkeDiagram.png}
	\caption{PakkeDiagram af #3}
	\label{fig:#2_PakkeDiagram}
\end{figure}
}

\newcommand{\KlasseDiagram}[3]{
\begin{figure}[H]
	\centering
	\includegraphics[width=#1\textwidth]{SoftwareArkitektur/#2/Diagrammer/#3_KlasseDiagram.png}
	\caption{KlasseDiagram af #3}
	\label{fig:#3_KlasseDiagram}
\end{figure}
}

\newcommand{\SekvensDiagram}[3]{
\begin{figure}[H]
	\centering
	\includegraphics[width=#1\textwidth]{SoftwareArkitektur/#2/Diagrammer/#3_SekvensDiagram.png}
	\caption{SekvensDiagram af #3}
	\label{fig:#3_SekvensDiagram}
\end{figure}
}

\newcommand{\StateDiagram}[3]{
\begin{figure}[H]
	\centering
	\includegraphics[width=#1\textwidth]{SoftwareArkitektur/#2/Diagrammer/#3_StateDiagram.png}
	\caption{StateDiagram af #3}
	\label{fig:#3_StateDiagram}
\end{figure}
}


\newcommand{\RevisionsTabel}[2]
{
\begin{table}[H]
\centering
{\rowcolors{2}{white!80!black!30}{white!70!black!60} %farver på hver anden række -starter på 3
\setlength{\arrayrulewidth}{0.2mm}					 %tykkelse på linier 
\setlength{\tabcolsep}{10pt}						 %indryk i celle 
\renewcommand{\arraystretch}{1.5}					 %højden på tabelrum
\center
\begin{tabular}{|>{\raggedright}p{4cm}|>{\raggedright}p{4cm}|>{\raggedright\arraybackslash}p{4cm}|}		 %længden på alle rum
\hline

\multicolumn{3}{|>{\columncolor{white!20!black!90}}m{13.44cm}|}{\textcolor{white}{\large{\textbf{Revision}}}} \\\hline
\rowcolor{white!70!black!60}
\textcolor{black}{\large{\textbf{Ændret af}}}&
\textcolor{black}{\large{\textbf{Version}}}&	
\textcolor{black}{\large{\textbf{Dato}}}\\
\hline
#2
\hline
\end{tabular}
}
\caption{Revision for #1}
\label{table:#1_Revision}
\end{table}
}

\newcommand{\systemSignaler}[3][FLAF]{
\begin{table}[H]
\centering
{\rowcolors{2}{white!80!black!30}{white!70!black!60} %farver på hver anden række -starter på 3
\setlength{\arrayrulewidth}{0.2mm}					 %tykkelse på linier 
\setlength{\tabcolsep}{10pt}						 %indryk i celle 
\renewcommand{\arraystretch}{1.5}					 %højden på tabelrum
\center
\begin{tabular}{|>{\raggedright}p{20mm}|>{\raggedright}p{40mm}|>{\raggedright}p{30mm}|>{\raggedright\arraybackslash}p{30mm}|}		 %længden på alle rum
\hline

\multicolumn{4}{|>{\columncolor{white!20!black!90}}m{14.20cm}|}{\textcolor{white}{\large{\textbf{Signal beskrivelser}}}} \\\hline
\rowcolor{white!70!black!60}
\textcolor{black}{\large{\textbf{Navn}}}&
\textcolor{black}{\large{\textbf{Definition}}}&	
\textcolor{black}{\large{\textbf{Område}}}&
\textcolor{black}{\large{\textbf{Kommentar}}}\\
\hline
#3
\hline
\end{tabular}
}
\ifthenelse{ \equal{#1}{FLAF} }{
\caption{Signal beskrivelser for #2}
\label{sig:#2_Signal}
}{
\caption{Signal beskrivelser for #2}
\label{sig:#1_Signal}
}

\end{table}
}

\newcommand{\accepttest}[3][FLAF]
{
\begin{table}[H]
\centering
{\rowcolors{2}{white!80!black!30}{white!70!black!60} %farver på hver anden række -starter på 3
\setlength{\arrayrulewidth}{0.2mm}					 %tykkelse på linier 
\setlength{\tabcolsep}{10pt}						 %indryk i celle 
\renewcommand{\arraystretch}{1.5}					 %højden på tabelrum
\center
\small
\begin{tabular}{|>{\raggedright}p{4.5cm}|>{\raggedright}p{4.5cm}|>{\raggedright\arraybackslash}p{4.5cm}|}		 %længden på alle rum
\hline

\multicolumn{3}{|>{\columncolor{white!20!black!90}}m{14.94cm}|}{\textcolor{white}{\large{\textbf{Accepttest #2}}}} \\\hline
\rowcolor{white!70!black!60}
\textcolor{black}{\large{\textbf{Test}}}&
\textcolor{black}{\large{\textbf{Forventet resultat}}}&	
\textcolor{black}{\large{\textbf{Resultat}}}\\
\hline
#3
\hline
\end{tabular}
}
\ifthenelse{ \equal{#1}{FLAF} }{
\caption{Acceptest for #2}
\label{table:#2_AcceptTest}
}{
\caption{Acceptest for #2}
\label{table:#1_AcceptTest}
}
\end{table}
}

% To change a font in the middle of a section
\newcommand*{\kodefont}{\fontfamily{pcr}\selectfont}
\DeclareTextFontCommand{\textKode}{\kodefont}
\newcommand{\funkArg}[2]
{
& \textKode{#1}	& \multicolumn{3}{p{115mm}}{#2}\\
}

\newcommand{\funk}[4]
{
\begin{table}[H]
\begin{tabular}{p{5mm}p{20mm}p{50mm}p{50mm}p{20mm}}
\multicolumn{5}{p{145mm}}{\textKode{#1}}\\
& \multicolumn{4}{p{140mm}}{#2}\\
& \textbf{Returnere} & \multicolumn{3}{p{150mm}}{#3}\\    
#4                          
\end{tabular}
\end{table}
}

%---------------------------CODE--------------------------
\usepackage{listings}
\definecolor{commentGreen}{RGB}{34,139,24}
\definecolor{stringPurple}{RGB}{208,76,239}

\lstset{language=Matlab,					% Sprog
	basicstyle=\ttfamily\scriptsize,		% Opsaetning af teksten
	keywords={for,if,while,else,elseif,		% Noegleord at fremhaeve
			  end,break,return,case,
			  switch,function},
	keywordstyle=\color{blue},				% Opsaetning af noegleord
	commentstyle=\color{commentGreen},		% Opsaetning af kommentarer
	stringstyle=\color{stringPurple},		% Opsaetning af strenge
	showstringspaces=false,					% Mellemrum i strenge enten vist eller blanke
	numbers=left, numberstyle=\tiny,		% Linjenumre
	extendedchars=true, 					% Tillader specielle karakterer
	columns=flexible,						% Kolonnejustering
	breaklines, breakatwhitespace=true,		% Bryd lange linjer
}
% skriv:
% %\lstinputlisting[language=c]{main.c}  	language: om det er c, vhdl ect.
%											filen: filen koden er i.
%-----------------------------------------------------------
