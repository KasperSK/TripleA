
%%%%%%%%%%%%%%%%%%%%%%%Opsætning af format%%%%%%%%%%%%%%%%%%%%%%%%%%
\documentclass[a4paper,oneside]{memoir} %A4papir, to side, størrelse 12, type memoir 	%TWOSIDE!!!!!!	
%\documentclass[a4paper,oneside]{memoir} %A4papir, to side, størrelse 12, type memoir 	%ONESIDE!!!!!!

% For dansk opsætning med æ, ø og å, samt pænere orddeling.
\usepackage[utf8]{inputenc}				% æøå
\usepackage[danish]{babel}				% dansk opsætning
\renewcommand{\danishhyphenmins}{22}	% fikser babel fejl/bedre orddeling
\usepackage[T1]{fontenc}
\usepackage{lmodern} 


%%Høre til under tabel (Pakker, men skal stå før pgfplot pga. default værdier)%%
\usepackage[table]{xcolor}	

\usepackage{graphicx} 				% Haandtering af eksterne billeder (JPG, PNG, EPS, PDF)
\usepackage{multirow}               % Fletning af raekker og kolonner (\multicolumn og \multirow)
\usepackage{wrapfig}				%for figure der skal have tekst om sig
\usepackage{float}					% Muliggoer eksakt placering af floats, f.eks. \begin{figure}[H]
\usepackage{enumerate}				% Muliggoer at man kan bruge fx a) i enumerate
\usepackage{pdflscape}				% Muligør landskab på enkelte sider
\usepackage{tabularx}				% Tabeller med X width
\usepackage{mathtools}				% Formler og matematik
\usepackage{pdfpages}				% til at inkludere PDF filer
\usepackage[footnote,draft,danish,silent,nomargin]{fixme}	% For at holde styr på mangler i teksten
\usepackage{ifthen}					% If then
\usepackage{hyperref}
\usepackage{textcomp,gensymb}				% symboler til latex
\usepackage{enumitem}				% Muligheder for ref til item

% ?? Using
%\usepackage[bottom]{footmisc} %fodnoter i bunden af arket

% Complains
% \usepackage{caption}

%--------------------------------------------------------------------
% Billeder bliver lagt i Media
%--------------------------------------------------------------------
\graphicspath{{../Media/}}

%--------------------------------------------------------------------
% Marginer
%--------------------------------------------------------------------
\usepackage{ragged2e,anyfontsize}		% Justering af elementer
\raggedbottom	%Ingen side strech for twopage
\setlrmarginsandblock{3cm}{3cm}{*}		%Højre - venstre
\setulmarginsandblock{3cm}{2.5cm}{*}	%Øverst - nederst
\checkandfixthelayout[nearest]    		%Specifikt valg af højde algoritme
% No more use 2015 \usepackage{fixltx2e}					% Retter forskellige fejl i LaTeX-kernen

%--------------------------------------------------------------------
%  Inholdsfortegnelse
%--------------------------------------------------------------------
\setcounter{tocdepth}{3} % inkludere sub + subsubsection i inholdsfortegnelde
\setsecnumdepth{subsection}


%--------------------------------------------------------------------
% Kapitel formatering
%--------------------------------------------------------------------
\usepackage{titlesec} 
\titleformat{\chapter}		  % Fjerne Kapitel og tal fra chapters
			{\Large\bfseries} % format
			{}                % label
			{0pt}             % sep
			{\huge}           % before-code

%--------------------------------------------------------------------
% Sektion formatering
%--------------------------------------------------------------------
\titlespacing\section{0pt}
{24pt plus 4pt minus 2pt}{6pt plus 2pt minus 2pt} %halvt mellemrum efter section

\titlespacing\subsection{0pt}
{18pt plus 4pt minus 2pt}{2pt plus 2pt minus 2pt} %kun lidt mellemrum efter subsection

\titlespacing\subsubsection{0pt}
{18pt plus 4pt minus 2pt}{2pt plus 2pt minus 2pt} %kun lidt mellemrum efter subsubsection 

\setlength\parindent{0pt} %Ingen indryk efter ny afsnit



%--------------------------------------------------------------------
% Sidehoved og -fod
%--------------------------------------------------------------------
\let\footruleskip\undefined  %fixer memoir default footruleskip
\usepackage{fancyhdr}
\pagestyle{fancy}
\fancyhf{}
\fancyhead[C]{\textit{Katrines Kælder Kasseapparat}}
\fancyfoot[RO,LE]{\thepage\ af \thelastpage}  	% sættet sidetal tal h/v efter om der er ulige
												% eller lige sidetal

\fancypagestyle{plain}{% bruges ved Undtagelser						
  \fancyhf{}%
  \fancyfoot[RO,LE]{\thepage\ af \thelastpage}	
  \renewcommand{\headrulewidth}{0pt}			%Ingen linje ved chapter, kun sidetal
}

%--------------------------------------------------------------------
% Literaturliste liste
%--------------------------------------------------------------------
\usepackage[danish]{varioref}				% Muliggoer bl.a. krydshenvisninger med sidetal (\vref)
\usepackage{natbib}							% Udvidelse med naturvidenskabelige citationsmodeller
\bibpunct[,]{[}{]}{;}{a}{,}{,} 				% Definerer de 6 parametre ved Harvard henvisning 
											% (bl.a. parantestype og seperatortegn)
%\bibliographystyle{../Latex/Litteratur/plainnat}		% Udseende af litteraturlisten.
\bibliographystyle{IEEEtran}

%--------------------------------------------------------------------
% Ordliste
%--------------------------------------------------------------------
\usepackage[nonumberlist,toc]{glossaries}
\makeglossaries
\newglossaryentry{betalingskort}{
	name=Betalingskort,
	description={Et kort som kan bruges som betalingsmiddel f.eks. et Dankort eller VISA/Dankort}
}

\newglossaryentry{betalingsterminal}{
	name=Betalingsterminal,
	description={En terminal der bruges til at udføre transaktioner, når et betalingskort bruges som betalingsmiddel}
}

\newglossaryentry{vare}{
	name=Vare,
	description={En vare kan bruges til at omtale en alkoholisk drik eller en øl}
}

\newglossaryentry{system}{
	name=Systemet,
	description={Katrines Kælders Kasseapperat}
}

\newglossaryentry{bartender}{
	name=Bartender,
	description={Den primære bruger af systemet}
}

\newglossaryentry{Administrator}{
	name=Admin,
	description={Brugeren som varetager systemet}
}

\newglossaryentry{gls-MVVM}{
	name=Model-View-ViewModel,
	description={En systemarkitektur mønster som bruges i brugergrænsefladen}
}
\newacronym{MVVM}{MVVM}{Model-View-ViewModel\glsadd{gls-MVVM}}

\newglossaryentry{brugergraenseflade}{
	name=brugergrænseflade,
	description={Grænseflade mellem bruger og system}
}
\newglossaryentry{gls-GUI}{
	name=Graphical User Interface,
	description={Se \Gls{brugergraenseflade}}
}
\newacronym{GUI}{GUI}{Graphical User Interface\glsadd{gls-GUI}}

\newglossaryentry{gls-WPF}{
	name=Windows Presentation Framework,
	description={Et framework der bruges til at præsentere brugergrænsefladen}
}
\newacronym{WPF}{WPF}{Windows Presentation Framework\glsadd{gls-WPF}}

\newglossaryentry{gls-XAML}{
	name=Extensible Application Markup Language,
	description={Et sprog der bruges til at deklarere brugergrænsefladen}
}
\newacronym{XAML}{XAML}{Extensible Application Markup Language\glsadd{gls-XAML}}

\newglossaryentry{gls-EF}{
	name=Entity Framework,
	description={Et framework der bruges til at tilgå databasen}
}
\newacronym{EF}{EF}{Entity Framework\glsadd{gls-EF}}

\newglossaryentry{gls-DAL}{
	name=Data Access Layer,
	description={En grænseflade mellem Entity Framework (databasen) og systemet}
}
\newacronym{DAL}{DAL}{Data Access Layer\glsadd{gls-DAL}}

\newglossaryentry{forretningslogik}{
	name=forretningslogik,
	description={Et reglsæt for systemets behandling af data}
}
\newglossaryentry{gls-BLL}{
	name=Business Logic Layer,
	description={Se \Gls{forretningslogik}}
}
\newacronym{BLL}{BLL}{Business Logic Layer\glsadd{gls-BLL}}

\newglossaryentry{View}{
	name=View,
	description={En grafisk brugergrænseflade}
}

\newglossaryentry{ViewModel}{
	name=ViewModel,
	description={En abstraktion mellem forretningslogikken og den grafiske brugergrænseflade}
}

\newglossaryentry{Model}{
	name=Model,
	description={Se \Gls{forretningslogik}}
}

\newglossaryentry{FluentAPI}{
	name=Fluent API,
	description={En API som i forhold til databasen binder objekt relationer til entity relationer}
}

\newglossaryentry{gls-API}{
	name=Application Programming Interface,
	description={En dokumenteret grænseflade til software komponenter}
}
\newacronym{API}{API}{Application Programming Interface\glsadd{gls-API}}

\newglossaryentry{gls-DTO}{
	name=Data Transfer Object,
	description={En datamæssig afkobling af system og database, hvor ens databasemodel overføres til et system objekt}
}
\newacronym{DTO}{DTO}{Data Transfer Object\glsadd{gls-DTO}}

\newglossaryentry{gls-REST}{
	name=Representational State Transfer,
	description={Et arkitekturmønster til at designe netværksapplikationer}
}
\newacronym{REST}{REST}{Representational State Transfer\glsadd{gls-REST}}

\newglossaryentry{ASPNET}{
	name=ASP.NET,
	description={Et web applikations framework}
}

\newglossaryentry{gls-Dao}{
	name=Data Access Object,
	description={En klasse som abstrahere \gls{DAL} i forhold til det emne, der omhandles}
}
\newacronym{Dao}{Dao}{Data Access Object\glsadd{gls-Dao}}

\newglossaryentry{controller}{
	name=controller,
	description={En klasse der styrer logikken for det emne, der omhandles}
}

\newglossaryentry{RESTful}{
	name=RESTful,
	description={Et API som opfylder \gls{REST} mønsteret}
}

\newglossaryentry{WebAPI}{
	name=WebAPI,
	description={En implementering af et \gls{RESTful} API}
}

\newglossaryentry{gls-DSD}{
	name=Data Structure Diagram,
	description={En diagram som beskriver data strukturer især i forhold til databasen}
}
\newacronym{DSD}{DSD}{Data Structure Diagram\glsadd{gls-DSD}}

\newglossaryentry{DDS-Lite}{
	name=DDS-Lite,
	description={Et program som bruges til at designe databaser}
}

\newacronym{IHA}{IHA}{Ingeniørhøjskolen Aarhus Universitet}
\newacronym{ASE}{ASE}{Aarhus University School of Engineering}



%--------------------------------------------------------------------
% Custom funktioner
%--------------------------------------------------------------------

% Forsiden
\usepackage{../Latex/Functions/frontpage}

% Usecases
\usepackage{../Latex/Functions/usecases}

% Actors
\usepackage{../Latex/Functions/actors}

% Sysml
\usepackage{../Latex/Functions/sysml}

% Accepttests
\usepackage{../Latex/Functions/accepttests}

\newcommand{\plus}{\raisebox{.4\height}{\scalebox{.6}{+}}}
\newcommand{\minus}{\raisebox{.4\height}{\scalebox{.8}{-}}}

\newcommand{\RevisionsTabel}[2]
{
\begin{table}[H]
\centering
{\rowcolors{2}{white!80!black!30}{white!70!black!60} %farver på hver anden række -starter på 3
\setlength{\arrayrulewidth}{0.2mm}					 %tykkelse på linier 
\setlength{\tabcolsep}{10pt}						 %indryk i celle 
\renewcommand{\arraystretch}{1.5}					 %højden på tabelrum
\center
\begin{tabular}{|>{\raggedright}p{2cm}|>{\raggedright}p{2cm}|>{\raggedright}p{2cm}|>{\raggedright\arraybackslash}p{6cm}|}		 %længden på alle rum
\hline

\multicolumn{4}{|>{\columncolor{white!20!black!90}}m{14.18cm}|}{\textcolor{white}{\large{\textbf{Revision}}}} \\\hline
\rowcolor{white!70!black!60}
\textcolor{black}{\large{\textbf{Ændret af}}}&
\textcolor{black}{\large{\textbf{Version}}}&	
\textcolor{black}{\large{\textbf{Dato}}}&
\textcolor{black}{\large{\textbf{Ændringer}}}\\
\hline
#2
\hline
\end{tabular}
}
\caption{Revision for #1}
\label{table:#1_Revision}
\end{table}
}

\newcommand{\systemSignaler}[3][FLAF]{
\begin{table}[H]
\centering
{\rowcolors{2}{white!80!black!30}{white!70!black!60} %farver på hver anden række -starter på 3
\setlength{\arrayrulewidth}{0.2mm}					 %tykkelse på linier 
\setlength{\tabcolsep}{10pt}						 %indryk i celle 
\renewcommand{\arraystretch}{1.5}					 %højden på tabelrum
\center
\begin{tabular}{|>{\raggedright}p{20mm}|>{\raggedright}p{40mm}|>{\raggedright}p{30mm}|>{\raggedright\arraybackslash}p{30mm}|}		 %længden på alle rum
\hline

\multicolumn{4}{|>{\columncolor{white!20!black!90}}m{14.20cm}|}{\textcolor{white}{\large{\textbf{Signal beskrivelser}}}} \\\hline
\rowcolor{white!70!black!60}
\textcolor{black}{\large{\textbf{Navn}}}&
\textcolor{black}{\large{\textbf{Definition}}}&	
\textcolor{black}{\large{\textbf{Område}}}&
\textcolor{black}{\large{\textbf{Kommentar}}}\\
\hline
#3
\hline
\end{tabular}
}
\ifthenelse{ \equal{#1}{FLAF} }{
\caption{Signal beskrivelser for #2}
\label{sig:#2_Signal}
}{
\caption{Signal beskrivelser for #2}
\label{sig:#1_Signal}
}

\end{table}
}



% To change a font in the middle of a section
\newcommand*{\kodefont}{\fontfamily{pcr}\selectfont}
\DeclareTextFontCommand{\textKode}{\kodefont}
\newcommand{\funkArg}[2]
{
& \textKode{#1}	& \multicolumn{3}{p{115mm}}{#2}\\
}

\newcommand{\funk}[4]
{
\begin{table}[H]
\begin{tabular}{p{5mm}p{20mm}p{50mm}p{50mm}p{20mm}}
\multicolumn{5}{p{145mm}}{\textKode{#1}}\\
& \multicolumn{4}{p{140mm}}{#2}\\
& \textbf{Returnere} & \multicolumn{3}{p{150mm}}{#3}\\    
#4                          
\end{tabular}
\end{table}
}

%---------------------------CODE--------------------------
\usepackage{listings}
\definecolor{commentGreen}{RGB}{34,139,24}
\definecolor{stringPurple}{RGB}{208,76,239}

\lstdefinestyle{customc}{
  belowcaptionskip=1\baselineskip,
  breaklines=true,
  frame=L,
  xleftmargin=\parindent,
  language=C,
  showstringspaces=false,
  basicstyle=\footnotesize\ttfamily,
  keywordstyle=\bfseries\color{green!40!black},
  commentstyle=\itshape\color{purple!40!black},
  identifierstyle=\color{blue},
  stringstyle=\color{orange},
}

\lstset{escapechar=@,style=customc}


% skriv:
% %\lstinputlisting[language=c]{main.c}  	language: om det er c, vhdl ect.
%											filen: filen koden er i.
%-----------------------------------------------------------