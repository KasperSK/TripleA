\chapter{Systembeskrivelse}

Ud fra systemet beskrevet i opgaveformuleringen er det besluttet at det endelige system består af en database, business logik, \gls{GUI}, \gls{WebAPI}. 
\newline
\newline
En bartender kan betjene det digitale kasseapparat via en \gls{GUI}. \gls{GUI}'en tager fat i business logikken, business logikken tager så fat i database logikken, der tilgår databasen. I databasen gemmes data, som for eksempel produkter, transaktioner, salg m.m.
Når en kunde ønsker at betale skal det kunne foregå via flere betalingsmåder, heriblandt f.eks. kontant, betalingskort, Mobilepay, m.m.
\newline
\newline
Hvis baren ønsker at ændre i eller tilføje data i databasen skal det være muligt for Administratoren at tilføje, redigere og slette disse gennem et Web interface. Ydermere skal der også være muligt for Administratoren at se statistik over alle salg, samt detaljeret statistisk over et enkelt salg. 

\newline
\newline
GUI'en skal designes således at den skal være så brugervenlig for Bartenderen som muligt. Bartenderen skal kunne gennemføre et køb på få klik. Bartenderen vælger de varer ind som ønskes og vælger derefter til den betalingstype, der skal betales med. Når et køb er gennemført skal dette gemmes i databasen, så statisk over salg kan vises og så de senere skal kunne tilgås. 
\newline
\newline
Kasseapparatets salgsfunktioner skal kunne tilgås via \gls{GUI}'en i form af en touchskærm. Det skal være muligt at tilgå indstillinger og statistik via et web interface, som bruger \gls{WebAPI}'en til at tilgå databasen, web interfacet tilgås via en browser. 
                  
\begin{figure}[h]
    \centering
    \includegraphics[width=0.8\textwidth]{Forside/Rigebillede}
    \caption{Rigebillede til illustration af systemet}
    \label{fig:rig_billede}
\end{figure} 

