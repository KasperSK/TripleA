%!TEX root = ../main.tex
\chapter{Projektafgrænsning}
Fra \gls{IHA}  er der på forhånd defineret nogle krav til projektets indhold, hvilket indebærer, at der skal:
\begin{itemize}
\item inddrages faglige aspekter fra samtlige fag på 4. semester IKT. 
\item anvendes en iterativ udviklingsproces
\item udvikles applikationer med grafiske brugergrænseflader, databaser og
netværkskommunikation.
\item anvendes teknikker, metoder og værktøjer til softwaretest
\item anvendes objektorienteret analyse og design i systemudvikling
\item anvendes projekt- og versionsstyringsværktøjer
\newline\newline
\end{itemize}

Ud fra overstående bygges et system, der kan administrere salg i en bar. 
Systemet skal kunne klare flere betalingsformer, det lykkedes dog ikke at implementere andet end kontant betaling, men systemet er sat op til at kunne udvides med flere betalingsformer. 
Det skal også være muligt for en administrerende bruger at tilføje/fjerne og redigere varer i systemet.  
\newline\newline
Der vil i projektet være en applikation med grafisk brugergrænseflade og via denne, kan en bartender tilgå varer i en database. Systemet, der kommunikeres igennem, vil være objektorienteret. 

Til at teste systemet vil der blive anvendt værktøjer og metoder fra faget softwaretest, såsom test frameworket Nunit, hvori der vil blive opsat unit- og integrationstest. 
\newline\newline
Det vil blive efterstræbt at implementere en fuldt funktionsdygtig prototype af systemet, 
men med overnævnte krav vil der kunne forekomme problemer med at få al funktionalitet implementeret. 
Derfor er der også skåret de steder, hvor det blev vurderet nødvendigt, herunder betalingsformer og mulighed for at printe bonner.
\newline\newline
Netværkskommunikation sker ved, at der er lavet et \gls{WebAPI}, hvor den administrerende bruger kan redigere varer og se statistik.  
\newline\newline
Som versionsstyringsværktøj er der anvendt Git, hvor der har været en master-branch, og de individuelle personer har lavet ændringer i deres personlige branch, som så bliver forenet med masteren, når opgaven er færdig.  
