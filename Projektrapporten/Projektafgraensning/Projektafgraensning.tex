%!TEX root = ../main.tex
\chapter{Projektafgrænsning}
Fra IHA  er der på forhånd defineret nogle krav til projektets indhold, hvilket indbærer:
\begin{itemize}
\item Projektet skal inddrage faglige aspekter fra samtlige fag på 4. semester IKT. 
\item Anvende en iterativ udviklingsproces
\item Udvikle applikationer med grafiske brugergrænseflader, databaser og
netværkskommunikation.
\item Anvende teknikker, metoder og værktøjer til softwaretest
\item Anvende objektorienteret analyse og design i systemudvikling
\item Anvende projekt- og versionsstyringsværktøjer
\newline\newline
\end{itemize}

Ud fra overstående bygges et system der kan administrere salg i en bar. Systemmet skal kunne klare flere betalingsformer og det skal være muligt for en administrerende bruger at tilføje/fjerne og redigere varer i systemmet.  
\newline\newline
Der vil i projektet være en applikation med grafisk brugergrænseflade og via denne, kan en bartender tilgå varer i en database. Systemet der kommunikeres igennem vil være objektorienteret. 

Til at teste systemmet vil der blive anvendt værktøjer og metoder fra faget softwaretest, såsom test frameworket Nunit hvori der vil blive opsat unit- og integrationstest. 
\newline\newline
Det vil blive efterstræbt at implementere en fuldt funktionsdygtig prototype af systemmet, men med overnævnte krav, vil der kunne forekomme problemer med at få al funktionalitet implementeret da tidsbegrænsningen er stor. 
\newline\newline
Netværkskommunikation sker ved at der er lavet en WebApi hvor den administrerende bruger kan redigere varer og se statisik.  
\newline\newline
Som versionsstyringsværktøj at der anvendt Git, hvor der har været en master-branch og de individuelle personer har lavet ændringer i deres personlige branch, som så bliver forenet med masteren når opgaven er færdig.  
