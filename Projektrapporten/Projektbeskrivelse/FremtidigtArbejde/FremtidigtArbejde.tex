%!TEX root = ../../main.tex
\section{Fremtidigt Arbejde}
Systemet er lavet med plads til udvidelser. Derfor vil der kontinuert kunne laves nye features og der er også plads til forbedrelser. I det følgende vil der blive beskrevet hvilke udvidelser og forbedrelser der kunne laves til systemet i fremtiden. 

\subsection{GUI}
GUI'en kom godt med fra start og fik implementeret mange af de ønskede features der skulle bruges i systemet. Den fik endda et ansigtsløft hvor den blev gjort pæn at se på og ikke kun funktionel.\newline 
\subsubsection{Print af kviteringer}
I dens nuværende form har GUI'en ikke en knap der printer kundernes kviteringer, og dette ville være det primære der skulle arbejdes på i forhold til GUI'en i fremtiden. I forhold til print, har GUI'en en knap til at printe afstemning af kasseapparatet, og der er i koden lavet et udkast til hvad der sker når man trykker på denne knap. Dette er imidlertid udkommateret, da der i business logikken ikke er implementeret så der kan udprintes afstemning.\newline
\subsubsection{TabView og -Model}
Noget helt andet der kunne forbedres i fremtiden, er det view og den ViewModel der hører til Tabs og produkter. Her er der i nuværende implementering ikke taget højde for hvis der er flere produkter af en type i databasen. I viewet kan der vises 25 produkter, men hvis der er flere går det ud over funktionaliteten. I fremtiden kunne man lave det sådan at hvis der er flere end 25 bliver den sidste knap en pil der bladrer til næste side i den følge produkttype. Således ville man have flere sider under samme produkttype.\newline
\subsubsection{SettingsView og -Model}
I fremtiden ville det være en fordel at sørge for at få implementeret et view kaldet SettingView. Dette var tiltænkt at være et view hvor man kunne styre opsætningen af selve programmet. F.eks. hvis nu brugeren af produktet såsom Katrines kælder skiftede navn, ville man her kunne ændre det, og så ville det have indflydelse på hvad der stod på en kvitering. Hvad mere er, at man kunne sætte op hvilken Database der skulle forbindes til. 

\subsection{Web}
Web \gls{API} er blevet implementeret hurtigt derfor er er også meget der kan forbedres.

\subsubsection{login system}
Der kunne med fordel laves et login system så det kun er administrator der kan til gå siden.

\subsubsection{Layout}
Siden kunne sættes bedre op og det bagved liggende javascript kunne optimeres.

\subsubsection{Statistik}
Der kunne tilføjes grafik til statistikken og mulighederne for at lave udtræk med bestemte parametre.

\subsection{Forretnings logikken}
I forretnings logikken er også rig mulighed til forbedring

\subsubsection{Betalings metoder}
Her kunne klart implementeres Dankort og mobile pay/swip. 
Samt alt hvad der skal bruges til at understøtte dette bland andet betaling med penge tilbage til kunden.

\subsubsection{Afstemning}
Det skal være muligt at printe afstemningen ud istedet for blot at vise denne på gui. 
