%!TEX root = ../../main.tex
\section{Opnåede erfaringer}
Gennem projektarbejdet er der arbejdet ud fra en iterativ udviklingsproces. Til processen er der lånt elementer fra Scrum, som f.eks. sprint møder, sprint retrospektiv og kanban board. Over projektperioden har der været syv sprint, hvor de iterative elementer har indgået. En problemstilling som opstod under de første sprints, var opgavens størrelse. Dette blev løst efterfølgende ved planlægning af det næste sprint, hvor opgavemængden blev reduceret.
\newline\newline
I de indledende sprints blev systemet designet efter nogle af de software designprincipper, som er erfaret gennem kurset Software Design. Her blev det besluttet, at systemet skulle være lagdelt efter 3-lags modellen, som skiller softwaren i: \textit{Presentation Layer}, \textit{Business Layer} og \textit{Data Layer}. Lagene blev uddelt i forskellige pakker. Dette sikrer, at koblingen mellem delene er lavere, hvilket gør testbarheden af de enkelte dele større, da delene kan agere enkeltvis.
\newline\newline
Systemarkitekturen er skrevet ud fra N+1 modellen. N+1 modellen er en blanding af arkitektur og design uden at gå i dybden med implementeringen. Det sørger for, at tankegangen bag implementeringen er dokumenteret på sådan en måde, at en udefrakommende kan sætte sig ind i systemet.
\newline\newline 
Til klasseimplementeringen blev der implementeret en testsuite, som skulle sørge for, at klassens metoder opfyldte dens kontrakter. Dette hjalp med at udradere nogle slåfejl, som ved integrering kunne have fået en test til at fejle og i sidste ende applikationen.
\newline\newline
Under alt udvikling blev der ført versionshistorik gennem Github. Dvs. at alle projektets ændringer i forhold til softwaren samt dokumentationen og rapporten blev fulgt. I forbindelse med vores repository var der sat en TeamCity CI server op, som sørgede for at de pull requests, som blev lavet, ikke fik tests til at fejle, før at de kunne tilføjes til masteren.
\newline\newline
Systemet havde en database til at gemme data. Data er f.eks. ordre, transaktioner og produkter. I første omgang blev databasen lavet ud fra Model First. Det viste sig, at det var nemmere at lave om i databasen vha. EF, hvilket betød at databasen blev bygget med Code First i stedet.
\newline\newline
Systemet havde også et web interface til administration. Denne er bygget med ASP.NET, som bruges til at generere en WebAPI til web interfacet. Dette betyder, at Administratoren vha. sin browser kan administrere kasseapparatet.