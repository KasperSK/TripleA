%!TEX root = ../../main.tex
\section{Opnåede Erfaringer}
I forbindelse med programmeringen af den grafisk grænseflade er der gjort nogle erfaringer omkring dynamisk opsætning.
HEr skulle der alt efter hvilke data der lå i databasen sættes et view op. Hertil kom der udfordringer lå i hvordan
man fik omsat denne til data til forskellige attributter i de forskellige controls. Ydermere opstod der et problem 
hvor skriftstørrelsen på knapper ikke ville ændre sig i takt med at hoved vinduet blev større. Dette blev løst med
i code behind med et event der blev kaldt når størrelsen på vinduet ændrede sig og derved satte skriftstørrelsen relativt 
til dette.

I forbindelse med arbejdsprocessen er der gjort en erfaring omkring det at arbejde iterativt. Hertil har det være svært at vurdere
opgavens omfang og her burde det omfanget være designet løbende i den iterative proces. 

I forbindelse med test af software er der gjort store erfaringer, da det er teste et så stort så system ofte belyser mange fejl
som man derved kan forebygge. Hertil har det været smart at bruge Continous Integration til udvikling af disse test, da sammen
med den iterative arbejdsproces har Continous integration gjort det muligt at bruge værktøjer fra faget software test løbende. 

Designet af systemet kunne gøres på mange måde men også her er der gjort en erfaring af hvad der i nogle tilfælde kunne være smart. 
Her er det erfaret at opdele i systemet i flere lag giver en større abstraktion og en lavere kobling og derved opnåes der en større 
testbarhed af systemet da hvert lag kan testes for sig. 

Til systemet skulle der også opsættes en database. Dette kunne gøres manuelt i SQL eller man kunne bruge Entity Framework til at skrive 
C# kode og opsætte databasen automatisk. Her er der nået en konklussion og gjort den erfaring at Entity Framwork gr arbejdet lettere, samtidig med
at det giver en stor overskuelighed omkring databasen. 
