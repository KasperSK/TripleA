%!TEX root = ../../main.tex
\section{Metoder}

I dette projekt er der blevet brugt følende arbejdsmetoder:
\begin{itemize}
	\item V-Modellen
	\item SCRUM
	
\end{itemize}

\subsection{V-Modellen}

V-Modellen, som er en udviklingsmodel, er blevet brugt i dette projekt til at lave test løbende. Da der laves test sideløbende sikres der at systemet virker efter hensigten. På figur \ref{VModel} kan V-Modellen ses.

\begin{figure}[H]
	\centering
	\includegraphics[scale=1.0]{Rapport/VModel.PNG}
	\caption{V-Modellen}
	\label{VModel}
\end{figure} 
Den første udviklingsfase er udarbejdelse af kravspecifikationen. Her udvikles der en tilhørende accepttest, der verificerer systemets overordnede funktionalitet. Den næste fase er systemarkitekturen.
I denne fase udvikles tests af integrationen mellem implementerede moduler. Under design og implementering som udgør den sidste fase i udviklingen udføres der løbende unittests, af de implementerede moduler.

\subsection{SCRUM}
Udviklingsmetoden har være Scrum inspireret, da proces ikke har været en fuld ud tilrettelagt er det ikke ren Scrum. En af grundene har været at processen bliver uforudsigelig, når der bruges nye værktøjer og derfor har der været behov for fleksible tidsplaner og hyppige gennemgange. 
\newline
\newline
I dette projekt er det blevet anvendt til at dele størrer opgave op i små dele til de forskellige sprint. Samt at holde daglige møder om arbejdet og opgaverne. 


