%!TEX root = ../../main.tex
\section{Design og Implementering Software}

\subsection{GUI}
\subsubsection{GUI Design}
Den grafiske brugergrænseflade er først og fremmest lavet ud fra et design som Katrines kælder
har ønsket. Dette design er baseret på det eksisterende kasseaparats tastatur og mærkater,
således at overgangen fra at skulle bruge det gamle kasseaparat til det nye ville mindskes.
\newline
\newline
Hertil er der blevet tegnet et skitse som kuhnne bruges til at lave et mock-up af den kommende GUI.
\newline
\newline
(Indsæt billede af skitse)
\newline
\newline
\subsubsection{Mock-up}
Det første mock-up blev lavet som en ren illustrativ GUI uden nogen form for funktionalitet. 
Hertil blev den lavet uden fornemmelse for hvad der var smart men kun med tankerne om 
hvordan den kunne ligne det gamle kasseaparats tastatur. 
\newline
\newline
(Indsæt udklip af Mock-up)
\newline
\newline
\subsubsection{MVVM}
Da det første mock-up var lavet gik der et sprint hvor der intet GUI blev lavet, men mere fokuseret
på business logik. Herefter blev der taget fat i GUI'en igen hvor selv hovedvinduet blev delt i tre 
UserContents. Dette var smart da det blev aftalt at GUI'en skulle laves efter MVVM princippet og her kunne
de tre usercontents så repræsentere et view hver. 
Til hver af disse views skulle der så laves en tilhørende viewModel som kunne hente data fra businesslogikken
således at Viewet ikke skulle tænke på dette selv. Den tilhørende viewmodel skulle nemlig modellere viewet alt efter 
hvad der skete i businesslogikken. Dette blev gjort med Bindings fra View til Viewmodel, Og Events fra businesslogik til 
Viewmodel. 
\newline
\newline
Selve udseendet af GUI'en blev præget meget af Resources der skulle gøre det hele mere ensartet. Hertil er der blevet dataTemplates, 
styles og bindings mm.
 \newline
 \newline
Den færdige GUI kom til i sin enkelthed at lige det mock-up der blev lavet, men der kom meget mere fokus på controls der var smarte, og 
hvilke der ikke var smarte og så fik den en fin afpusning der gjorde den pænere at se på.  