%!TEX root = ../main.tex
\chapter{Resumé}
\subsubsection*{Dansk}

Denne rapport beskriver semesterprojektet for 4. semesterprojektet på IHA, retning IKT. Problemstillingen omhandler designet og implementeringen af et kasseapparat der er nemt at betjene også når bartendere har fået lidt indenbords. Projektoplægget var givet af Katrines kælder og derfor er dette projekt designet med henblik på et kasseapparat til 
katrines kælder der er brugervenligt for bartendere. 
\newline
\newline
Dette realiseres ved at en bartender betjener kasseapparatet ved hjælp af en \gls{GUI} designet i \gls{WPF}. Via denne \gls{GUI} bliver der via businesslogikken kommunikeret med en database. Ved hjælp af denne \gls{GUI} kan en bartender via et kasseapparat udføre salg og retur af varer. Ved hjælp af en designet  \fxnote{gls web api} Web Api kan en administrator redigere i databasen. 
\newline
\newline
Det samlede produkt vil indeholde et kasseapparat med en tilhørende \gls{GUI}  der kan printe bon og betjenes med kontantbetaling, med mulighed for udvikling af andre betalings metoder og en Web Api \fxnote{gls} der kan tilføje, redigere samt fjerne produkter, produktgrupper, produkttyper og produktfaner i databasen. . 
\newline
\newline
Udviklingsprocessen er gennemført efter en scrum tidplan, det vil sige iterativt. I dette tilfælde har det bestået af 2 uger lange sprints. I disse iterationer er dokumentation samt rapport blevet udarbejdet. Dokumentationen er struktureret efter N+1 modellen \fxnote{gls på n+1}. Løbende som udviklingenprocessen er nået mod ende er arbejdet på og ændret i dokumentationen og rapporten ud fra de opnåede erfaringer et projektforløb har givet. Ud fra dette arbejde er der opnået et gennemarbejder system samt rapport og dokumentation. 
\newline
\newline
\newline
\newline

\chapter{Abstract}

\subsubsection*{English}

This report describes the 4th semester project at ASE, with IKE as the field of study. 