%!TEX root = ../main.tex
\chapter{Resumé}
\subsubsection*{Dansk}

Denne rapport beskriver et 4. semesters projekt på Ingeniørhøjskolen på Aarhus Universitet, studieretning Informations- og Kommunikationsteknologi. 
Problemstillingen for projektet omhandler design og implementering af et digitalt kasseapparat der er nemt at betjene også selvom bartender har fået lidt indenbords. 
Oplægget til projektet er givet af Katrines Kælder som er fredagsbaren for ingeniørerne på Katrinebjerg Campus.
\newline
\newline
For at løse problemstillingen implementeres der en \gls{GUI} designet i \gls{WPF}. Via denne 
\gls{GUI} er lavet til at være brugervenlig og let at betjene. 
Samtidigt laves der også et forretnings logik lag som skal tage sig af alt der har med salg og lignende at gøre. 
Det hele gemmes i en database der også kan tilgås igennem et Web interface således at der kan redigeres i produkter og lignende. 
Det gør at en bartender kan bruge kasseapparatet til salg og en administrator kan tilføje/fjerne produkt samt se hvilke salg der har været foretaget.
\newline
\newline
Det færdige produkt vil være et digitalt kasseapparat med en brugervenlig \gls{GUI} 
der kan printe bonner og betjenes med kontantbetaling, med mulighed for udvikling af andre betalingsmetoder. 
Til dette er også en web API der gør det muligt at fjerne/tilføje produkter, produktgrupper, produkttyper og produkttabs i database. 
Derudover kan der også ses hvilke salg der har været på web interfacet.
\newline
\newline
Udviklingsprocessen er udført iterativt. 
Der er blevet gjort brug af elementer fra SCRUM og der blev arbejdet i sprint med varighed af 2 uger. 
I hver iteration er dokumentation samt rapport blevet udarbejdet. 
Dokumentationen er struktureret efter N+1 modellen og den er løbende blevet opdateret ud fra de erfaringer der blev gjort i løbet af sprintene.   
\newline
\newline
















