%!TEX root = ../main.tex
\chapter{Indledning}
På \gls{IHA} ligger fredagsbaren Katrines kælder som er samlingssted for de studerende når timerne 
fredag eftermiddag ebber ud. Her står der fredag efter fredag flittige og frivillige medarbejdere som gør deres bedste for at 
tilfredsstille de studerendes behov for kolde drikkevarer. Disse kunder betaler betaler ofte med dankort. Dette er i sig selv et krævende job, men når der oveni ligger arbejde
med at skrive afstemning og man ved hvert salg skal huske at trykke dankort på selve kasseapparatet men også på dankort 
terminalen kan der hurtigt tjenes sorte penge, hvilket baren ikke interesseret i. 
Her har fredagsbaren udstedt et råb om hjælp og dette råb er blevet hørt. 
\newline
\newline
Formålet med dette semesterprojekt er at lave et digitalt kasseapparat der kan lette bartenders byrde når han fredag eftermiddag/aften langer drikkevarer over disken i Katrines kælder. Et kasseapparat hvor bartenderen kan sælge vare fra kælderen igennem en GUI, mens informationer om alle salg bliver gemt i en database, som så kan tilgås via en browser.
\newline
\newline
Katrines Kælder har lavet et opslag hvor de fortæller at de ønsker et digitalisere deres kassesystem, som skal være nemt at bruge, selv i stresset situationer. Da dette projekt har krav til at læringsmålene skulle opfyldes og at projektet skal inddrage faglige aspekter fra samtlige fag på 4. semester IKT, er kasseapparatet til Katrines kælder et godt projekt da der her kan argumenteres for at alle fag er indraget på hver deres måde.
\newline
\newline
Disse problemer er løst ved at lave en GUI svarende til Katrines Kælders nuværende kasseapparat, hvor denne kan gemme og hente salg fra en database. Hertil er der lagt op til at Dankort ligger side om side med andre betalingsmetoder så man ikke skal trykke to steder når der skal betales med dankort. 
\newline
\newline
Med udgangspunkt i Katrines kælders behov er der opstillet en række Use Cases, der beskriver aktøres interaktion med systemet. Ud fra kravspecifikationen er der dannet design af software til systemmet der er dokumenteret med N+1 arkitektonisk view model. Der bliver ydermere også præsenteret de arbejdsmetoder der anvendt til at gennemføre projektet. Til sidst i rapporten ses resultater og diskussion, samt et afsnit om muligt fremtidigt arbejde. 