%!TEX root = ../main.tex
\chapter{Indledning}

På Ingeniørhøjskolen i Århus, ligger fredagsbaren og samlingsstedet Katrines Kælder.
Her kan de studerende og ansatte få sig en kold øl eller drink når fyratensklokken ringer.
Her står bartenderne klar til at tage imod de tørstige folks ordrer og tager selv engang imellem
en lille tår for at blive forfrisket. 
\newline
Til rådighed har bartenderne et arsenal af forskellig alkohol, spil såsom kort og terninger, musikanlæg
og et kasseapparat der kan registrere ordrer. Alt synes at skulle være tip top, men selvom beliggenheden
er en ingeniørhøjskole der underviser i elektricitet og software er kasseapparatet der er det vigtigste 
modul i baren, uddateret og ikke smart. Aparatet består nemlig af et keyboard med påklistrede mærkater og 
kun små skærme til at følge med i den igangværende ordre. Knapperne er små og kan være svære at ramme når man 
som bartender har fået lidt at drikke. 

formålet med dette projekt er udvikle et moderne kasseaparat der skal kunne gøre det lettere for bartenderne når 
de fredag aften skal 
servicere de tørstige kunder. 
\newline

Hertil skal der udvikles en database indeholdende de forskellige produkter som baren kan tilbyde.
\newline


Endvidere skal tilkobles businesslogik der kan håndtere den igangværende ordre og elementer i databasen. 
\newline


Ydermere skal produktet have en tilhøre touchskærm hvortil der laves en GUI som bartenderne kan navigere igennem 
når en kunde skal serviceres. GUI'en skal bruges knapper som hver repræsenterer et produkt som så kan ende i selve 
ordren når der trykkes betal.
\newline 


Produktet skal også have adgang til en bonprinter som skal printer skal printer bonen for den igangværende eller forhenværende 
ordre efter ønske. 
\newline


Som betalingsmiddel skal produktet kunne tage imod kontanter, dankort og om muligt mobilaplikkationer såsom Mobile Pay. 
Kontanter skal kunne lægges i en kontantskuffe som åbnes når ordren er gennemført. 
\newline

Med udgangspunkt i Katrines Kælders anvisninger og krav, er der blevet opstilt Use Cases der beskrives forløbet af forskellige
ordrer der gennemføres. 











   