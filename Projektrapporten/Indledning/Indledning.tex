%!TEX root = ../main.tex
\chapter{Indledning}

Projektet er udført på 4. semester, hvor kurserne har dannet grundlag for den viden, der har været nødvendig for udførelsen af dette projekt. Kravet til projektets indhold er givet af et oplæg på klassen inden projektets start. Produktet var valgfrit dog med det krav at alle semestrets fag skulle indkluderes samt et versionsstyrings værktøj.
\newline
\newline
Selvom projektet var frit, var der et par projektoplæg man kunne vælge. Katrine Kælders projektoplæg, som var et Digitalt kasseapparat, blev valgt. 
\newline
\newline
Projektformuleringen havde disse fokuspunkter:

\begin{itemize}
  \item Systemet er i stand til at kommunikere med dankortterminalen i forhold til at sende beløb til dankortterminalen og modtage godkendt/afvist.
  \item Brugerinterfacet indeholder de nødvendige knapper (tilsvarende det gamle kasseapparat)
  \item Kasseopgørelserne skal gemmes på en database
\end{itemize} 
 
Katrines Kælder oplever i travle tider at deres kasseapparat er en flaskehals når det gælder betjening af kunder. Derfor ønsker de at både at digitalisere og opdatere men også effektivisere deres nuværende kasseapparat. Dette vil resultere i hurtigere betjening og derfor kortere ventetid for Katrine Kælders gæster.
\newline
\newline
Tanker inden projekt er at lave en GUI svarende til Katrines Kælders nuværende kasseapparat, denne skal kommunikere med en database for at gemme salg mm. for at mindske Katrine Kælders papirarbejde. System skal være pålideligt og simpelt for bartendere at overskue og benytte.
\newline
\newline
Udarbejdelsen af dette produkt vil ske ved udarbejdelse af en problemstilling, som løses igennem procesorienteret gruppearbejde. De forskellige arbejdsopgaver vil blive fordelt mellem gruppemedlemmerne. Der bliver både udarbejdet en projektdokumentation, samt en projektrapport.
\newline
\newline
I rapporten vil der beskrives de valg og overvejelser, der er har gjort i projektforløbet, samt redegøre for de konsekvenser dette medfører. Rapporten vil ligeledes indeholde en simplificeret gennemgang af,
hvordan de forskellige dele i projektet fungerer og er udarbejdet.
\newline
\newline
Projektdokumentationen vil indeholde alle de tekniske informationer, samt en detaljeret gennemgang af projektets
dele fra designfasen til det færdig produkt.










   