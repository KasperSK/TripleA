%!TEX root = ../main.tex
\chapter{Indledning}
[indledning 1]\newline
Projektet er udført på 4. semester, hvor kurserne har dannet grundlag for den viden, der har været nødvendig for udførelsen af dette projekt. Kravet til projektets indhold er givet af et oplæg på klassen inden projektets start. Produktet var valgfrit dog med det krav at alle semestrets fag skulle indkluderes samt et versionsstyrings værktøj.
\newline
\newline
Selvom projektet var frit, var der et par projektoplæg man kunne vælge. Katrine Kælders projektoplæg, som var et Digitalt kasseapparat, blev valgt. 
\newline
\newline
Projektformuleringen havde disse fokuspunkter:

\begin{itemize}
  \item Systemet er i stand til at kommunikere med dankortterminalen i forhold til at sende beløb til dankortterminalen og modtage godkendt/afvist.
  \item Brugerinterfacet indeholder de nødvendige knapper (tilsvarende det gamle kasseapparat)
  \item Kasseopgørelserne skal gemmes på en database
\end{itemize} 
 
Katrines Kælder oplever i travle tider at deres kasseapparat er en flaskehals når det gælder betjening af kunder. Derfor ønsker de at både at digitalisere og opdatere men også effektivisere deres nuværende kasseapparat. Dette vil resultere i hurtigere betjening og derfor kortere ventetid for Katrine Kælders gæster.
\newline
\newline
Tanker inden projekt er at lave en GUI svarende til Katrines Kælders nuværende kasseapparat, denne skal kommunikere med en database for at gemme salg mm. for at mindske Katrine Kælders papirarbejde. System skal være pålideligt og simpelt for bartendere at overskue og benytte.
\newline
\newline
Udarbejdelsen af dette produkt vil ske ved udarbejdelse af en problemstilling, som løses igennem procesorienteret gruppearbejde. De forskellige arbejdsopgaver vil blive fordelt mellem gruppemedlemmerne. Der bliver både udarbejdet en projektdokumentation, samt en projektrapport.
\newline
\newline
I rapporten vil der beskrives de valg og overvejelser, der er har gjort i projektforløbet, samt redegøre for de konsekvenser dette medfører. Rapporten vil ligeledes indeholde en simplificeret gennemgang af,
hvordan de forskellige dele i projektet fungerer og er udarbejdet.
\newline
\newline
Projektdokumentationen vil indeholde alle de tekniske informationer, samt en detaljeret gennemgang af projektets
dele fra designfasen til det færdig produkt.

\newpage
[indledning 2]\newline
På ingeniørhøjskolen i Aarhus ligger fredagsbaren Katrines kælder som er samlingssted for de studerende når timerne 
fredag eftermiddag ebber ud. Her står der fredag efter fredag flittige og frivillige medarbejdere som gør deres bedste for at 
tilfredsstille de studerendes behov for kolde drikkevarer. Disse kunder betaler betaler ofte med dankort. Dette er i sig selv et krævende job, men når der oveni ligger arbejde
med at skrive afstemning og man ved hvert salg skal huske at trykke dankort på selve kasseapparatet men også på dankort 
terminalen kan der hurtigt tjenes sorte penge, hvilket baren ikke interesseret i. 
Her har fredagsbaren udstedt et råb om hjælp og dette råb er blevet hørt. 
\newline
\newline
Formålet med dette semesterprojekt er at lave et digitalt kasseapparat der kan lette bartenders byrde når han fredag eftermiddag/aften langer drikkevarer over disken i Katrines kælder. Et kasseapparat hvor bartenderen kan sælge vare fra kælderen igennem en GUI, mens informationer om alle salg bliver gemt i en database, som så kan tilgås via en browser.
\newline
\newline
Katrines Kælder har lavet et opslag hvor de fortæller at de ønsker et digitalisere deres kassesystem, som skal være nemt at bruge, selv i stresset situationer. Da dette projekt har krav til at læringsmålene skulle opfyldes og at projektet skal inddrage faglige aspekter fra samtlige fag på 4. semester IKT, er kasseapparatet til Katrines kælder et godt projekt da der her kan argumenteres for at alle fag er indraget på hver deres måde.
\newline
\newline
Disse problemer er løst ved at lave en GUI svarende til Katrines Kælders nuværende kasseapparat, hvor denne kan gemme og hente salg fra en database. Hertil er der lagt op til at Dankort ligger side om side med andre betalingsmetoder så man ikke skal trykke to steder når der skal betales med dankort. 
\newline
\newline
Med udgangspunkt i Katrines kælders behov er der opstillet en række Use Cases, der beskriver aktøres interaktion med systemet. Ud fra kravspecifikationen er der dannet design af software til systemmet der er dokumenteret med N+1 arkitektonisk view model. Der bliver ydermere også præsenteret de arbejdsmetoder der anvendt til at gennemføre projektet. Til sidst i rapporten ses resultater og diskussion, samt et afsnit om muligt fremtidigt arbejde. 