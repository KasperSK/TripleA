%!TEX root = ../../main.tex

\section{OverordnedeSekvensDiagrammer}
I dette afsnit vises de overordnede sekvensdiagrammer som er lavet med det formål at kunne vise interaktionen mellem aktør og systemet. Samtidig viser der også systemets funktionalitet i henhold til kravspecifikationen, og derfor at gøre det lettere at forstå systemet i forhold til den ydre verden. 


\subsection{Overordnet sekvensdiagram for Use Case 1 ''Kontant salg''}
Diagrammet viser interaktion mellem Bartender og system for Use Case 1, ''Kontant Salg'' 

\sysml{0.6}{SEQ}{UseCase1}{Use Case 1}

Som der kan ses på på sekvensdiagrammet figur \ref{fig:UseCase1_SEQ}, har bartenderen mulighed for at opdaterer betalings listen inde i et loop, indtil alle bestillingerne er tastet ind. For at komme ud af loopet skal Bartenderen enten annullerer købet eller gå til Betaling. 


\subsection{Overordnet sekvensdiagram for Use Case 2 ''Udskriv kundebon''}
Diagrammet viser interaktion mellem Bartender og system for Use Case 2, ''Kontant Salg'' 

\sysml{0.6}{SEQ}{UseCase2}{Use Case 2}

Når Bartender printer en Bon kan skal systemmet udskrive en bon, men i tilfælde af at der ikke er mere papir vil systemmet melde en besked om at der ikke er mere papir.
\newline\newline
For at se hvordan man skifter Bon papiret skal man se under Use Case 2 - Extension 2.1.



\subsection{Overordnet sekvensdiagram for Use Case 3 ''Kasseafstemning''}
Diagrammet viser interaktion mellem Bartender og system for Use Case 2, ''Kasseafstemning'' 

\sysml{0.6}{SEQ}{UseCase3}{Use Case 3}

Når Bartender skal afstemme kassen afslutter han Use Casen med udskiver en udskrive en bon over afstemningen, men i tilfælde af at der ikke er mere papir vil systemmet melde en besked om at der ikke er mere papir.
\newline\newline
For at se hvordan man skifter Bon papiret skal man se under Use Case 3 - Extension 3.2.