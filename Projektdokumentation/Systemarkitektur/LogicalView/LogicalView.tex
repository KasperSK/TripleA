\section{Logical View}
Dette kapitel beskriver systemets opdeling i delsystemer. Her ser vi på de funktionaliteter som systemmet giver til brugeren. 

\subsection{Oversigt}

\subsection{Arkitektursignifikante designpakker	}
\subsubsection{Pakke 1: GUI}

\subsubsection{Pakke 2: Business Layer}
I dette afsnit beskrives koden i projektets business layer
\subsubsection{Sales}
%\logicalview{0.85}{CLASS}{Sales}{pakken Sales}



\subsubsection{Orders}
%\input{Systemarkitektur/LogicalView/BusinessLayer/Order}

\subsubsection{Payment}
%\logicalview{1}{CLASS}{Payment}{pakken Payment}
\textbf{Payment pakken}
\newline
Payment pakken som ses på figur \ref{fig:Payment_CLASS}, er lavet for at kontrollere betalingen. Derfor er der lavet et interface \texttt{IPaymentProviderDescriptor} med beskrivelser. Fra den arver \texttt{IPaymentProvider}, som indeholder alle de forskellige metoder alle former for betaling typisk kommer igennem, som de forskellige betalingsformer, \texttt{CashPayment}, \texttt{Nets} og \texttt{MobilePay} er implementeret med. 
\newline
\newline
\texttt{PaymentController} har en liste af \texttt{IPaymentProvider}, hvor den så kan bruge deres metoder. På den måde har det kun være nødvendigt at lave en \texttt{PaymentControlleren} til alle slags PaymentProviders og ydermere er det forholdsvis simpelt at tilføje en ekstra form for betalingstype.
\newline
\newline
\texttt{PaymentControlleren} har til opgave at eksekvere transaktionen, gemme det i databasen og holde styr på opgørelsen. 

\subsubsection{Products}
%\input{Systemarkitektur/LogicalView/BusinessLayer/Product}

\subsubsection{CashDrawers}
%\textbf{CashDrawers pakken}
\newline
Denne pakke er lavet til at kunne styre pengeskuffen på systemet. Den er meget simpel og indeholder kun en funktion: \texttt{Open}, som navnet antyder bruges den til at åbne pengeskuffen.
I systemets nuværende tilstand skriver \texttt{CashDrawer} til loggen, at der er blevet bedt om at få skuffen åbnet.
I Figur~\ref{fig:CashDrawers_CLASS} kan det ses, at det er nemt at skifte \texttt{CashDrawer} ud med en implementation, der kan åbne den rigtige pengeskuffe, da det er implementeret med et interface.

\logicalview{0.35}{CLASS}{CashDrawers}{pakken CashDrawers}


\subsubsection{Receipts}
%\logicalview{0.85}{CLASS}{Receipts}{pakken Receipts}



\subsubsection{Printer}
%\textbf{Printer pakken}
\newline
Printer pakken er lavet for at gøre det muligt for programmet at udskrive kundebonner samt at udskrive afstemningen af kassen.
Til dette er der lavet et interface \texttt{IPrinter} dette er lavet for at gøre det nemt at udskifte printer implemtationen.
Interfacet er lavet til at modtage en linje ad gangen da vi tænkte at en bon er opbygget af ordrelinjer.
Selve klassen \texttt{ReceiptPrinter} er lavet til at kunne printe via windows printer dialogen.  

\logicalview{0.55}{CLASS}{Printer}{pakken Printer}


\subsubsection{Models}
%\logicalview{0.35}{CLASS}{Models}{pakken Models}



\subsubsection{Log}
%\textbf{Log pakken}
\newline
Pakken er implementeret for at ''logge'' altså udskrive handlinger i programmet til enten en konsol eller en tekstfil.
Ud fra et interface \texttt{ILogger}, som deklarer metoder til udskrivning af forskellige typer af beskeder, er interfacet implementeret i klassen \texttt{Logger}.
Klassen \texttt{Logger} indpakker frameworket \textit{log4net}\fxnote{skriv til ordliste}, som gør det muligt at ændre recipienten af beskederne om det skal være konsollen eller en tekstfil. Klassen \texttt{Logger} kan fås vha. \textt{LogFactory}, ved at kalde \texttt{GetLogger()}. Dette gør \texttt{CashDrawer} f.eks., da der ikke er implementeret en pengeskuffe på nuværende tidspunkt. Klassen udskriver i stedet en besked, når \texttt{Open()} bliver kaldt. Et klassediagram af \texttt{Log} pakken kan ses i Figur~\ref{fig:Log_CLASS}.

\logicalview{0.85}{CLASS}{Log}{pakken Log}

\subsubsection{Database}
%\logicalview{0.35}{CLASS}{Database}{pakken Database}

\subsubsection{Dal}
%\logicalview{0.35}{CLASS}{Dal}{pakken Dal}



\subsubsection{Pakke 3: Data}
Database designet er en to delt affære. Der er selve strukturen i databasen også er der koden som tilgår databasen.
Disse beskrive begge i følgende afsnit.

\subsubsection{Database Opbygning}
Opbygningen af den fysiske database er sket ved at opstille designet via et DS diagram som kan ses på figur \ref{fig:DSD} og derefter sætte den op med SQL Serber Database Project fra visual Studio. 

\begin{figure}[H]
    \centering
    \includegraphics[width=0.6\textwidth]{N+1/DataView/DabDSD}
    \caption{Data Structure Diagram over databasen}
    \label{fig:DSD}
\end{figure}
Databasen er lavet så det er muligt at have flere salgsbestillinger under et salg og et salg skal kunne betales med flere Transaktioner, disse er lavet ved at opsætte nogle one-to-many og many-to-many forhold, som der kan læses nærmere om under DataView i dokumentationen.
\newline
\newline
som det kan ses består OrderLine i databasen af alle de informationer en salgsbestilling kan bestå af, som vare, rabat og antallet af vare. mange af disse har deres egene tabeller så varer og rabatter også kan have nogle fast defineret værdier.
\newline
\newline
Som det kan ses på figur \ref{fig:DSD}, har den tabellerne Product, ProductGroup, ProductType og ProductTab. Dette design er lavet for at kunne give nogle større grupper rabat. ProductTaps er lavet i forhold til GUI'en så man nemt kunne få alle de vare der skulle stå på hver Taps ind.  


\subsubsection{Database kode}
For at kunne tilgå databasen fra C\# blev det besluttet at bruge \gls{EF}. 
Det blev valgt at bruge "Code First from existing database" da designet af databasen var lavet i forvejen.
Ved at følge den fremgangsmåde bliver der lavet en model af databasen i form af objekter som \gls{EF} 
kan mappe til i figur \ref{fig:CodeFirstFromDB} er princippet illustreret.

\begin{figure}[H]
    \centering
	\includegraphics[scale=1]{Rapport/EFCFFDB.PNG}
	\caption{Skitse af GUI}
	\label{fig:CodeFirstFromDB}
\end{figure} 

Det virkede meget godt til første iteration af databasen, men herefter viste det sig at være lettere at få \gls{EF} til at bygge
databasen udfra koden ved ændringer.
\newline
Det blev også besluttet at pakke \gls{EF} ind i \gls{DAL} dette blev brugt for at skille framework koden fra vores program.
\gls{DAL} Består af tre klasser en Facade klasse, en Unitofwork klasse og en Repository klasse. 
Facaden er brugt til at sørge for at der kun findes et Unitofwork af gangen. 
Unitofwork er en samling af Repositorys, Unitofwork's opgave er at sørge for at der bliver kaldt "Save" på \gls{EF}'s context.
Repository er metoder til at udføre Create, Read, Update and Delete.


\subsection{Use Case realiseringer	}
\subsubsection{ Use Case 1. realisering	}
\textbf{De moduler der realiserer Use Case 1 : Kontant salg}

\begin{enumerate}
	\item GUI
	\item SalesController
	\item OrderController
	\item OrderDao
	\item DalFascade
	\item UnitOfWork
\end{enumerate}

Use Case 1 bliver realiseret først og fremmest af GUI'en hvor bartenderen klikker på produkterne der skal sælges. GUI'en bruger så SalesControlleren Som bruger en orderController der tilføjer produkter i ordren. Når der skal betales bruges GUI'en SalesController som så kalder OrderControlleren som så kalder OrderDao som via DalFascade og unitOfWork tilgår databasen.  


\subsubsection{ Use Case 2. realisering	}
\textbf{De moduler der realiserer Use Case 2 : Udskriv kundebon}

\begin{enumerate}
	\item GUI
	\item SalesController
	\item ReceitController
	\item Receit
	\item ReceitPrinter
\end{enumerate}

Use Case 2 bliver realiseret først og fremmest af GUI'en hvor der efter et salg kan vælges at udskrive en bon over salget. Gui'en bruger SalesControlleren der kalder ReceitControlleren og laver en ny Receit som den udskriver via ReceitPrinter. 

\subsubsection{ Use Case 3. realisering}
\textbf{De moduler der realiserer Use Case 3 : Kasseafstemning}

\begin{enumerate}
	\item SalesController
	\item PaymentController
	\item GUI
\end{enumerate}

Use Case 3 bliver realiseret først om fremmest gennem SalesControlleren der via paymenControlleren får hentet hvor mange kontanter der er i kassen, og hvor meget der er blevet solgt med hver slags betalingstype, her vises det så på GUI'en. 


\subsubsection{Use Case 9. realisering}
\textbf{De moduler der realiserer Use Case 9: Tilføj varer til kasseapparat}

\begin{enumerate}
	\item Web API
	\item Database
	\item GUI
\end{enumerate}

Use Case 9 bliver realiseret primært gennem Web API'et som her bruger Database Controllere til at tilgå databasen. I Web API'et kan der indsættes nye produkter og det er samme database som den Business logikken henter i, hvortil der vises på GUI.


\subsubsection{Use Case 10. realisering}
\textbf{De Moduler der realiserer Use Case 10: Rediger varer i kasseapparat}

\begin{enumerate}
	\item Web API
	\item Database
	\item GUI
\end{enumerate}

Use Case 10 bliver realiseret primært gennem Wep API'et som her bruger Database Controllere til at tilgå databasen. I Wep API'et kan der redigeres i et allerede eksisterende produkt. Når et produkt er ændret som det ønskes, gemmes det i databasen, of da det er den samme database som business logikken henter fra, kan GUI'en vise det.

\subsubsection{Use Case 11. realisering}
\textbf{De Moduler der realiserer Use Case 11: Fjern varer fra kasseapparat}

\begin{enumerate}
	\item Web API
	\item Database
	\item GUI
\end{enumerate}
	
Use Case 11 bliver realiseret primært gennem Wep API'et som her bruger Database Controllere til at tilgå databasen. I Web API'et kan der fjernes allerede eksisterende produkter fra databasen. Når et produkt er fjernet opdateres databasen, og da det er den samme database som den business logikken henter fra, vil produktet ikke længere vises på GUI. 

\subsubsection{Use Case 12. realisering}
\textbf{De moduler der realiserer Use Cas 12: Vis statistik over salg}

\begin{enumerate}
	\item Web API
	\item Database
	\item GUI
\end{enumerate}

Use Case 12 bliver realiseret primært gennem Wep API'et som her tilgår Databasen via database Controllere. I Web API'et kan man vælger at se en statistik over de salg der foretaget. Disse kan være created, complete eller cancelled. For at der skal være noget at se statistik over, er det vigtig at man fra GUI'en har skabt en ordre. 



