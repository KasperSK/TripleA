Data pakken er lavet for at facilitere tilgang til databasen. Den er opdelt i tre pakker. En pakke, Models pakken, der har alle de klasser, som er nødvendige for at have data i hukommelsen. Den næste pakke er Database pakken, som står for tilgang og opsætning af databasen. Den sidste pakke er \gls{DAL}, som giver programmet mulighed for at tilgå databasen igennem pakken.
\newline\newline

\textbf{Models}\newline
Models pakken bliver brugt som model for den data der gemmes i databasen. Pakken indeholder en objekt repræsentation af data'en, der skal persisteres. Det er ikke en 1:1 repræsentation fordi \gls{EF} kan aflede en masse ud fra modellen. Desuden bruger vi \gls{FluentAPI}'en til at diktere forhold og nøgler i databasen.

\logicalview{0.85}{CLASS}{Models}{pakken Models}

I Figur~\ref{fig:Models_CLASS} ses alle klasserne i modellen, det vigtigste er at relationerne fremstår tydeligt mellem alle klasserne. Der kan ses en større sammenhæng i dataview sektionen\fxnote{reference}.
\newline

\textbf{Database}\newline
Database pakken indeholder selve logikken, der tilgår databasen. Kort fortalt er det her, vi implementerer den anden del af det, vi skal bruge i forhold til \gls{EF}, nemlig database konteksten. 
Den er vist i Figur~\ref{fig:Database_CLASS}, at der i denne pakke ligger instruktioner om, hvordan databasen hænger sammen. De ligger i \texttt{EntityConfiguration} klassen, som der er mange af. Derfor er de vist som en generisk klasse.
Klasserne bliver kaldt, når \texttt{CashRegisterContext} kalder \texttt{OnModelCreating()}. Konfigurationen fortæller om hvordan objekterne skal mappes til entiteterne i databasen f.eks. hvilken property som skal være primær nøgle, hvilke properties er påkrævet eller hvilke relationer objekterne har til hinanden. 

\logicalview{0.85}{CLASS}{Database}{pakken Database}
\textbf{Data Access Layer}\newline
\gls{DAL} pakken er lavet for at afkoble databasen fra programmet. 
Således at det kun er \gls{DAL}, der skal laves om, hvis database skiftes ud. 
I \gls{DAL} stilles alle de funktioner, der er nødvendige for at indsætte, slette, opdatere og oprette elementer i databasen til rådighed. 
Funktionerne kommer fra \texttt{Repository} klassen, som tilgås igennem \texttt{UnitOfWork} klassen.
\texttt{UnitOfWork} eksiterer for at sikre, at vi gemmer alle ændringerne i databasen på en gang.
Til sidst er der klassen \texttt{DALFacade}, som er en facade, der sørger for, at der kun kan eksistere et \texttt{UnitOfWork} ad gangen. 

\logicalview{0.85}{CLASS}{Dal}{Data Access Layer}

I Figur~\ref{fig:Dal_CLASS} kan sammenhængen ses. Her ses det, at \texttt{DalFacade} klassen er det, som vi udstiller til resten af programmet. Denne sørger for at give et \texttt{UnitOfWork} til dem, der har behov for at tilgå databasen.