\textbf{Order pakken}\newline
Order pakken er lavet for at holde styr på den nuværende ordre, der er igang samt de ordre som er lagt til side til senere.
Pakken stiller et interface \texttt{IOrderController} til rådighed, som definerer det, der skal til for at styre en ordre.
Klassen \texttt{OrderController} implementerer \texttt{IOrderController} og bruger \texttt{IOrderDao} til at tilgå databasen.
Når der skal tilføjes et product til en ordre kan \texttt{AddProduct()} kaldes. Metoden gemmer så produktet i en \texttt{OrderLine}, som den tilknytter den igangværende ordre.
\newline\newline
I figur \ref{fig:Orders_CLASS} ses det, at \texttt{OrderController} også implementerer interfacet \texttt{INotifyPropertyChanged}. Det er implementeret for at have muligheden for callbacks. Altså hvis en ordre er blevet ændret, skal brugergrænsefladen meddeles om ændringen.
Klassen \texttt{OrderDao} er et ''data access object''. Det eneste, som klassen foretager sig er at kalde databasen.

\logicalview{0.85}{CLASS}{Orders}{pakken Orders}