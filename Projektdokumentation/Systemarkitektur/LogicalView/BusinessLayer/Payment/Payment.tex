\textbf{Payment pakken}
\newline
\texttt{Payment} pakken, som ses på Figur~\ref{fig:Payment_CLASS}, er lavet for at kontrollere betalingen. For at have beskrivelser til betalingsmetoderne  er lavet et interface \texttt{IPaymentProviderDescriptor}. \texttt{IPaymentProvider} arver fra \texttt{IPaymentProviderDescriptor}. \texttt{IPaymentProviderDescriptor} er en klasse som indeholder de metoder, som alle betalingsmetoder kommer igennem. \texttt{Payment} klassen er lavet så alle betalingstyper -- \texttt{CashPayment}, \texttt{Nets} og \texttt{MobilePay} -- arver fra den abstrakte klasse \texttt{PaymentProvider}. Her er de forskellige PaymentProviders' metoder implementeret så de passer til betalingstypen. 
\newline\newline
Betalingstyperne har en metode, der overfører pengene, tilføjer det til det totale beløb i kassen og returnere status for transaktionen. Klasserne har også \texttt{Restart()} og \texttt{Shutdown()} metoder, som skal sørge for genstarte og afslutning af betalingsmetoderne. På nuværende tidspunkt er det kun \texttt{CashPayment}, der har en funktionalitet. \texttt{Nets} og \texttt{MobilePay} er kun implementeret, så der skrives til \texttt{Log}, når deres metoder bliver kaldt.  
\newline\newline
\texttt{PaymentController} har opgaven at udføre transaktioner, gemme informationer om transaktioner i databasen og holde styr på opgørelsen. \texttt{PaymentController} har en liste af \texttt{IPaymentProvider}. Når den skal udføre en transaktion, kan den aflæse på \texttt{Transaktion}, hvilken \texttt{PaymentType} den har. Ud fra dette bliver det afgjort hvilken \texttt{PaymentProvider} der skal anvendes. Når transaktionen er gennemført, bliver detaljerne om den gemt i databasen via \texttt{PaymentDao}. 

\logicalview{1}{CLASS}{Payment}{pakken Payment}