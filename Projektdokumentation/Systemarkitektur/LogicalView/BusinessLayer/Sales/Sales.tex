\logicalview{0.85}{CLASS}{Sales}{pakken Sales}

\newpage Pakken Sales, indeholder et interface og to klasser. Interfacet hedder ISalesController og er det interface som GUI'en bruger til at kommunikere med business logikken.
Den første klasse, SalesController implementerer iSalesController. Denne er den overordnede klasse som holder styr på businesslogikken. Denne bruges til at starte salg og hente produkter fra databasen som i sidste ende både udformer GUI'en og skaber et godt link mellem bruger og system. Tanken bag SalesControlleren er at den skal indeholde lav niveau moduler som den uddeligere opgaver til og derved skabe et løst forhold til de egentlige opgaver der bliver udført i systemet. På figur \ref{fig:Sales_CLASS} kan følgende klasse ses:

\begin{itemize}
	\item \textbf{SalesController: } skal håndtere salg
	\item \textbf{SalesFactory: } skal oprette en SalesController ud fra parametre
\end{itemize}