Pakken Sales indeholder ét interface og to klasser. Interfacet hedder \texttt{ISalesController}, og er det interface, som brugergrænsefladen bruger til at kommunikere med forretningslogikken. Klassen \texttt{SalesController} implementerer \texttt{ISalesController}. Dette er den overordnede klasse som holder styr på forretningslogikken. Tanken bag \texttt{SalesController} er, at den skal bruge de abstraktioner, som lav niveau modulerne implementerer. Klassen uddelegerer så opgaver til modulerne og derved skabes der et løst forhold til de egentlige opgaver, der bliver udført i systemet. 
\newline\newline
På Figur~\ref{fig:Sales_CLASS} kan følgende klasse ses. Her skal:
\begin{itemize}
	\item \texttt{SalesController} skal håndtere salg
	\item \texttt{SalesFactory} skal oprette en SalesController ud fra parametre
\end{itemize}

\logicalview{0.85}{CLASS}{Sales}{pakken Sales}