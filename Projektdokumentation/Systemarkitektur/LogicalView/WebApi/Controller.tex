For at kunne tilg� data fra databasen i vores web interface har vi implementeret et REST api.
Til dette har vi brugt en ASP.NETs indbyggede facilitteter \texttt{ApiController} ved at nedarve denne skal vi blot 
implementere \texttt{Get}, \texttt{Put}, \texttt{Post} og \texttt{Delete} funktionere i vores controllere.

\Diagram{0.85}{CLASS}{WebApi/Controllers}{Web controllers}{LogicalView}

Som det kan ses i figur \ref{fig:WebApi/Controllers_CLASS} er der mange controllere der er faktisk en til alle de typer data vi �nsker
at kunne hente ud fra databasen.
Desuden bruger alle Controllerne noget kaldt et \texttt{DTO} som st�r for \texttt{Data transfer object}, disse bliver brugt for at afkoble
database fra brugeren af data.

\newline
\textbf{Models}


I figur \ref{fig:WebApi/Controllers_MODEL} ses \texttt{DTO'erne} disse er blot klasser der kun indeholder data.

\Diagram{0.75}{MODEL}{WebApi/Controllers}{Web controllers}{LogicalView}