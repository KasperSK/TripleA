\subsubsection{Use Case 10: Rediger varer i kasseapperat}
Modulerne der realiserer UC10:
\begin{enumerate}
	\item CashRegisterConfig
	\item App.js
	\item HomeController
	\item ProductsController
	\item ProductTabController
	\item ProductTypesController
	\item ProductGroupController
	\item CashRegisterContext
\end{enumerate}

UC10 bliver realiseret primært gennem Web API'et. Dette sker via \textit{Settings View}. Når \textit{Settings View} bliver startet loades alle produkterne in i \texttt{App.js} ligesom i UC9. Derefter kan ændres i et produkt ved at trykke \textit{Details} på det pågældende produkt. Herefter kan der ændres i produktet. Når der er ændret i produktet, trykkes der \textit{Submit} og ændringerne gemmes i databasen. Forløbet kan ses i Figur~\ref{fig:UC10_SEQ}.

\Diagram{1.0}{SEQ}{UC10}{Use Case 10}{LogicalView}