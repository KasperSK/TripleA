\subsubsection{Use Case 11: Fjern varer fra kasseapparat}
Modulerne der realiserer UC11:
\begin{enumerate}
	\item CashRegisterConfig
	\item App.js
	\item HomeController
	\item ProductsController
	\item ProductTabController
	\item ProductTypesController
	\item ProductGroupController
	\item CashRegisterContext
\end{enumerate}
	
UC11 realiseres primært gennem Web API'en. Ligesom i UC9 og UC10 indlæses produkterne i \texttt{App.js}. Herefter kan der trykkes \textit{Delete} på produktet, som ønskes slettet. Dette laver et opslag på produktet i \texttt{App.js}, som sender en forespørgelse til sletning videre til \texttt{ProductsController}. \texttt{ProductsController} vil efterfølgende forespørge en sletning i \texttt{CashRegisterContext}. \texttt{CashRegisterContext} sletter produktet i databasen. Forløbet kan ses i Figur~\ref{fig:UC11_SEQ}.

\Diagram{1.0}{SEQ}{UC11}{Use Case 11}{LogicalView}