\section{Ikke fully dressed use cases}

\subsubsection*{Use Case 4 - Betalingskort Salg}
Skal beskrive forløbet for et salg når kasseapparatet interagerer med betalingsterminalen.

\subsubsection*{Use Case 5 - Rabat Salg}
Her i beskrives forløbet for salg, hvor der opnåes rabat på alle udvalgte varer.

\subsubsection*{Use Case 6 - Indtastning af lagervarer}
Til lagerstyrings delen af produktet skal der være mulighed for at indtaste varer i systemet når de ankommer/bestilles.

\subsubsection*{Use Case 7 - Administrator login}
For at kunne ændre priser, oprette ny salgsvarer i kasseapperatet osv. skal der kunne logges ind på en administrationsdel af systemet således at kun Administratorer kan gøre dette.

\subsubsection*{Use Case 8 - Fjern lagervarer}
Når der hentes varer på lageret skal disse kunne tjekkes ud af lager systemet.

%\subsubsection*{Use Case 9 - Tilføj varer til kasseapparat}
%Det skal være muligt at tilføje en vare til kasseapparatet denne use case kræver at administrator login, use case 7, er udført.

%\subsubsection*{Use Case 10 - Rediger varer i kasseapparat}
%Det skal være muligt at ændre en varepris og navn mm. denne use case kræver administrator login, use case 7, er udført.

%\subsubsection*{Use Case 11 - Fjern varer fra kasseapparat}
%Det skal være muligt at fjerne varer fra systemet hvis de f.eks. ikke længere bliver serveret. Denne use case kræver at administrator login, use case 7, er udført.

%\subsubsection*{Use Case 12 - Vis Statistik over salg}
%Der skal kunne genereres en statistik over de varer der er blevet solgt, gerne i detaljer der specifikt beskriver hvilke varer der er blevet solgt.

\subsubsection*{Use Case 13 - Scan varer til lager}
Et apparat kan bruges til at scanne scanne varer ind i lagersystemet. 

\subsubsection*{Use Case 14 - MobilPay/Swipp Salg}
Skal beskrive forløbet for et salg der betales via MobilPay eller Swipp.