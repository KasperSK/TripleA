%!TEX root = ../../main.tex

\subsection*{Use Case 3}
Denne Use Case beskriver hvordan der udskrives afstemning af kassen forløber ved udprintning af bon med alt der er solgt i løbet af aftenen.
\begin{usecase}{3}

\title{ Kasseafstemning } 

\field{Mål:}{ At Bartenderen har en bon med samlede salg og dato udskrevet }

\field{Initieret af:}{ Bartender }

\field{Aktører:}{ Bartender (Primær) }

\field{Samtidige forekomster:}{1}

%Preconditions: What must be true on start and worth telling the reader?
\field{Prækondition:}{System skal være tændt og klar}

%Postconditions: What must be true on successful completion and worth telling the reader
\field{Postkondition:}{ At bonen med det samlede salg er udskrevet }

%Main Success Scenario: A typical, unconditional happy path scenario of success.
\scenario{Hovedscenarie:}{
	\item Bartender vælger Indstillinger knappen
	\item System viser Indstillinger menu
	\item Bartender trykker på Afstem knappen
	\item System viser Afstem popup \newline
	[Extension 3.1: Bartender trykker på tilbage knappen]
	\item Bartender trykker på Godkend knappen 
	\item Systemets printer udskriver bonen \newline
	[Extension 3.2: Der er ikke mere papir]
	\item System viser Hovedmenuen
}

%Extensions: Alternate scenarios of success or failure.
\scenario{}{
	\item Extension 3.1: Bartender trykker på tilbage knappen
		\begin{enumerate}
			\item[1.] Systemet går tilbage til punkt 2
		\end{enumerate}
	\item Extension 3.2: Der er ikke mere papir
		\begin{enumerate}
		\item[1.] Systemets printer stopper. Dialogboks popper op med en besked om at skifte bonrulle.
		\item[2.] Bartender skifter bon i printer.
		\item[3.] Bartenderen trykker på OK knappen og use casen forsættes fra punkt 4
		\end{enumerate}
}


\end{usecase}